%!TeX root=MemoriaTFG.tex
\chapter{Arquitectura de la aplicación} \label{chapter:AppArchitecture}
En esta sección se presenta la arquitectura de la aplicación. Se realiza una 
descripción de las funcionalidades que lo componen, se definenen los requerimientos 
necesarios 
para su construcción y, por último, una explicación de la implementación, mostrando 
las herramientas usadas. El objetivo de este capítulo es definir al lector como está 
estructurada la aplicación, tanto a nivel de código desgranando las librerías externas de 
las que se ha hecho uso, como de la solución de la infraestructura, apostando por una 
solución que facilite al individuo su uso dentro del sistema.

\section{\textit{track-simulator}}
\textit{track-simulator} es un generador \textit{pseudo-aleatorio} de trayectorias 
geoposicionales. Se entiende como simulación al proceso de recrear el 
comportamiento, en este caso el camino de usuarios, dentro de un espacio 
geográfico. Como muestra esencial y discreta del comportamiento de la población se 
usan registros que han sido obtenidos mediante localización \ac{GPS}. En la figura 
\ref{figure:TrackExample1} se muestra un ejemplo de una detección del recorrido de un 
individuo.
\begin{figure}[!htb]
\begin{center}
\includegraphics[width=0.5\textwidth]{./Imagenes/RealTrackDetection.png}
\caption{Ejemplo de detección real de la trayectoria de un individuo por la Serra de 
Tramuntana, Illes Balears, Mallorca, Spain}
\label{figure:TrackExample1}
\end{center}
\end{figure}
\newpage

Para realizar un algoritmo que genere trayectorias y tengan un grado de similitud con la 
realidad se debe realizar un ejercicio previo de análisis y tratamiento de información. 
Distancias entre las detecciones, desviación entre la posición de una detección del 
dispositivo \ac{GPS} y un camino definido son ejemplos de datos que se tienen que 
tener en cuenta para la obtención de resultados óptimos en cuanto a representación de 
la realidad.

Tanto la información muestral como la generada, así como todo el conjunto de 
información geoespacial debe ser almacenada en un tipo de estructura lógica que
permita el acceso y manipulación de estos de la forma más eficiente posible. 

Teniendo una estructura lógica que permita almacenar la información, el problema 
determinante será la forma en la que los datos geoposicionales, en este caso puntos 
\ac{GPS}, pasan a formar parte de esté modelo. Recordemos que un punto \ac{GPS} 
contiene únicamente información de la posición dentro de la superficie terrestre, no 
obstante, no aporta información de la asignación de este punto a un segmento de
un camino, calle, o paraje concreto. La asociación de un punto \ac{GPS} a un tipo vía es 
un problema conocido con el nombre de \textit{Map matching} y la propuesta de este 
documento a su resolución se realizará en detalle en el apartado \ref{section: 
MapMatching}.

Con la estructura lógica y la información integrada en ella, el análisis de la información 
permitirá tomar el conjunto de decisiones que permitan maximizar el grado de exactitud 
de la simulación para, posteriormente, realizar el algoritmo que permita realizar dicha 
recreación. La transformación del dato desde una trayectoria real hasta la generación 
final de trayectorias simuladas queda ilustrada en la figura 
\ref{figure:TrackSimulatorDiagram}.
\begin{figure}[!htb]
\begin{center}
\includegraphics[width=0.5\textwidth]{./Imagenes/Structure}
\caption{Diagrama de funcionamiento de \textit{track-simulator}}
\label{figure:TrackSimulatorDiagram}
\end{center}
\end{figure}
\newpage
 
La aplicación crea una sucesión de puntos que equivalen a un recorrido dentro de un 
plano geográfico. Se ha determinado que la generación de puntos dentro del aplicativo 
se realiza de forma \textit{pseudo-aleatoria}, quiere decir que la producción de los 
diferentes puntos no sigue ningún patrón o regularidad en sí misma sino que parte del 
análisis realizado, al que se le añaden componentes aleatorios. La sucesión de puntos 
está relacionada directamente con un camino, sendero o recorrido asignado al modelo 
lógico. La simulación tiene en cuenta la frecuencia de paso relativa por el segmento 
como parte del proceso, así como la distancia entre los puntos uno a uno. Con este 
procedimiento se pretende obtener de una forma rigurosa un análisis cuantitativo de 
los datos aportados por las trayectorias reales para poder crear una simulación lo más 
real posible.

\section{Funcionalidades de \textit{track-simulator}}
\textit{track-simulator} es una aplicación. El término \textit{aplicación} se entiende 
como la suma de conjunto de implementaciones  de código ejecutable y del que se 
espera un resultado. El flujo del aplicativo se puede ver en la figura 
\ref{figure:TrackSimulatorDiagramFlow}. Consta de un único tipo de entrada, en este 
caso trayectorias sobre un terreno en ficheros \ac{GPX}. Los dos grandes módulos 
representan el análisis y la simulación de los datos, donde en el primero realizará una 
alimentación de una base de datos con la información de las trayectorias y el resultado 
del análisis, por otra parte almacenará en la máquina una imagen del resultado del 
análisis. El módulo de simulación realizará el proceso de generación de trayectorias a 
partir de la lectura de los datos almacenados en la base de datos, del que también se 
obtienen imágenes de las trayectorias.
\begin{figure}[!htb]
\begin{center}
\includegraphics[width=0.99\textwidth]{./Imagenes/DiagramEng}
\caption{Diagrama de funcionamiento de \textit{track-simulator}}
\label{figure:TrackSimulatorDiagramFlow}
\end{center}
\end{figure}

La aplicación recopila la siguiente serie de funcionalidades:
\begin{description}
\item [Importación y análisis de archivos \ac{GPX}] La aplicación soportará la 
importación de un archivo GPX concreto, o bien de un conjunto de archivos GPX. Estos 
archivos GPX deben contener trayectorias realizadas en el espacio delimitado. La 
funcionalidad del aplicativo se basa en un territorio acotado, por lo que el conjunto de 
rutas a analizar deben corresponder con el territorio especifico para que el proceso de 
análisis se complete. 

La aplicación, con los datos importados por los archivos, permitirá un análisis de 
trayectorias aplicando técnicas de \textit{map matching}. Del análisis se obtendrá 
información:
\begin{description} 
\item[Distancia entre puntos]  Distancia entre las capturas de posición \ac{GPS} en el 
fichero punto a punto.
\item[Distancia relativa entre punto de ruta y punto de trayectoria] Distancia del punto 
\ac{GPS} detectado de la trayectoria la punto proyección dentro de la ruta, camino o vía 
seleccionada como más probable en el proceso de \textit{map matching}.
\item[Frecuencia de paso por segmento de ruta] Frecuencia de paso por el segmento 
$x_{a}, x_{b}$ relativo a todos los segmentos desde $x_{a}$.
\end{description}
El detalle de esta información se describe en detalle en el capítulo 
\ref{section:ExplotacionDato}. Toda esta información quedará almacenada en una base 
de datos de forma que la información sea accesible para realizar la simulación de la 
ruta.
\item [Muestra de resultados del análisis] La aplicación permite la representación 
gráfica de información gráfica de los resultados del análisis. Las  imágenes serán 
almacenadas en formato \ac{PNG}.
\item [Creación de una trayectoria a partir de parámetros y exportación \ac{GPX}] Con 
los resultados de la información analizada se puede realizar una simulación de 
trayectorias dentro del espacio geográfico delimitado. Los parámetros a introducir 
quedan detallados en el capítulo \ref{chapter:GuiaUso}. Estas trayectorias quedarán 
representadas en formato \ac{GPX}. 
\item [Visualización de la trayectoria] La aplicación mostrará una representación gráfica 
de la trayectoria tanto   dentro del territorio geográfico a partir del almacenamiento de 
imágenes en formato  \ac{PNG}.
\end{description}


\section{Requerimientos de \textit{track-simulator}} \label{section: 
RequerimientosTrackSimulator}
Explicadas las funcionalidades de la aplicación \textit{track-simulator}, los 
requerimientos identificados son los siguientes:
\begin{enumerate}[label={R.\arabic*.}]

\item \textbf{Importación de datos} La aplicación debe realizar una importación de los 
datos desde un fichero \ac{GPX} al modelo lógico elegido para la el posterior 
tratamiento.

\item \textbf{Análisis de trayectoria} La aplicación debe realizar un análisis de la 
trayectoria que proporcione por una parte indicadores y por otro valores medibles, 
cuantificables y utilizables para realizar una simulación 
lo más precisa posible.

\item \textbf{Almacenamiento de las rutas reales en base de datos} La aplicación debe 
realizar un almacenamiento de los datos analizados en una base de datos, con el 
objetivo de poder ser repetible
sin necesidad de importar nuevamente los datos.

\item \textbf{Almacenamiento del análisis en base de datos} La aplicación debe realizar 
un almacenamiento de los resultados del análisis en una base de datos, con el objetivo 
de poder ser accesibles para la
etapa de simulación.

\item \textbf{Generación de fichero de gráficas resultantes del análisis}  La aplicación 
debe poder realizar una muestra de la información resultante para su visualización por 
parte del usuario.

\item \textbf{Lectura de la base de datos para el acceso a la información} La aplicación 
debe poder realizar una lectura de la base de datos para acceder a la información 
necesaria para la muestra de resultados o 
simulación de trayectorias.

\item \textbf{Simulación de puntos \ac{GPS} a partir de resultado de análisis} La 
aplicación debe poder generar el camino más probable dada un análisis previo y unos 
parámetros de entrada. La aplicación debe generar los puntos de cada uno de los 
segmentos necesarios dado un camino.

\item \textbf{Generación del fichero GPX correspondiente a la simulación de la 
trayectoria}  La aplicación realizar una exportación de los datos a formato \ac{GPX}.

\item \textbf{Generación de fichero visualización de trayectoria}  La aplicación debe 
poder mostrar al usuario una visualización de la trayectoria dado un fichero \ac{GPX}.
\end{enumerate}

\section{Implementación de track-simulator}
\subsection{Flujo de datos}
\begin{figure}[!htb]
\begin{center}
\includegraphics[width=0.9\textwidth]{./Imagenes/DataFlow}
\caption{Flujo de datos de 	\textit{track-simulator}}
\label{figure:TrackSimulatorDataFlow}
\end{center}
\end{figure}
Podemos ver en la figura \ref{figure:TrackSimulatorDataFlow} de la parte superior la 
implementación de la aplicación dos grandes módulos. Por una parte encontramos el 
módulo \textit{TrackAnalyzer}. Este módulo es el encargado de realizar todas los 
procesos de tratamiento de datos para su posterior análisis. Inicialmente en el alcance 
de este proyecto se realiza el análisis de ficheros \ac{GPX} por lo que es el único 
modelo de datos que entrará en nuestro sistema. Para realizar el análisis se realiza un 
proceso de tratamiento previo de datos, que se detalla en el apartado \ref{section: 
ImportacionGPX}.

Posterior al tratamiento de datos se realizará el proceso de \textit{map matching} con el 
que se une los datos introducidos al modelo lógico del territorio geográfico. El proceso 
de \textit{map matching} queda descrito en detalle en el apartado \ref{section: 
MapMatching}.

Una vez se ha realizado la asignación por cada punto a un camino determinado se 
puede realizar un análisis del fichero, del cual se obtienen las métricas requeridas por 
el apartado \ref{section: RequerimientosTrackSimulator}. La descripción de la forma en 
la que se explotan los datos está detallada en la sección \ref{section:ExplotacionDato}.

Con la explotación del dato hecha, únicamente queda guardar la información generada 
en una base de datos. El almacenamiento de las trayectorias analizadas y mapeadas, el 
resultado de los análisis y la estructura lógica con la información modificada se 
realizarán de forma independiente en secciones diferentes de la base de datos.

El módulo \textit{TrackSimulator} tiene la responsabilidad de generar la simulación de 
una trayectoria a partir de unos parámetros de entrada. Una vez generada la 
simulación, un fichero \ac{GPX} con las coordenadas y una representación gráfica de la 
trayectoria serán los productos finales del proceso.
 
\subsection{Herramientas externas utilizadas} 
Como herramientas externas utilizadas para la realizar la propuesta de este documento 
se destacan el tipo de almacenamiento de datos que se utilizarán, así como las librerías 
externas que se han usado para la implementación de la aplicación. A continuación se 
detallan ambas.

\subsubsection{Almacenamiento del dato}
Tanto de trayectorias reales como de los datos generados por el análisis deben ser 
almacenados de forma que sean accessibles de forma rápida.
Para el almacenamiento de datos, existen dos grandes posibilidades: el uso de Bases 
de datos relacionales, a partir de ahora descritas como \ac{SQLDB}, o el opuesto, las 
bases de datos no relacionales, \ac{NOSQLDB}.
Para la realización de la propuesta descrita en este documento se ha decidido realizar 
el almacenamiento de toda la información en \textit{MongoDB}, una de las 
\ac{NOSQLDB} más usadas y que detallamos a continuación.

\textit{MongoDB} se trata de una \ac{NOSQLDB} de código abierto y su funcionamiento 
es documental. 
Se almacenan colecciones de documentos, que son series de elementos clave-valor en 
formato JSON. 

MongoDB elimina las limitaciones de las bases de datos relacionales \cite{Mongo01}. 
Permite almacenar de forma eficiente grandes cantidades de información, siendo
a su vez flexible a modificaciones debido a que los documentos que se almacenan en 
las colecciones no tienen 
una taxonomía definida. Por lo que el desarrollo incremental de la aplicación y la 
aparición de nuevos 
campos dentro del modelo de datos no es un problema.
Otro gran motivo para la elección de MongoDB como base de datos de este proyecto 
es la escalabilidad que ofrece, de forma que a grandes cantidades de trayectorias por 
almacenar, podría distribuirse utilizando servicios en la nube.

Actualmente para el desarrollo de la propuesta no se ha contado con una gran cantidad 
de datos. No obstante el problema ha sido planteado con el objetivo de poder analizar 
grandes cantidades de datos y que puedan ser almacenados y accesible mediante el 
uso de esta base de datos.

\subsubsection{Librerías utilizadas} \label{subsection: LibreriasExternas}
El lenguaje seleccionado para el desarrollo del aplicativo ha sido Python al ser 
recomendable para el tratamiento de datos. Su flexibilidad y la gran utilidad aportada 
por librerías externas hacen que sea idóneo para el desarrollo de la propuesta. 
Entendemos como librería un conjunto de código agrupado que aporta funcionalidad 
diversa. Las librerías externas empleadas en este proyecto son las siguientes :
\begin{enumerate}[label={L.\arabic*.}]
%gpxpy
\item \textbf{gpxpy} Librería para la manipulación de ficheros \ac{GPX}. Mediante esta 
librería se puede realizar una importación del fichero para su posterior manipulación. 
De esta forma se obtiene la secuencia de puntos \ac{GPS} que describe una trayectoria 
determinada.
%pandas
\item \textbf{pandas} Librería para el análisis de los datos. Toda la información 
correspondiente a las diferentes trayectorias queda reflejada en un Dataframe para su 
posterior tratamiento y acceso.
%osmnx
\item \textbf{osmnx} Librería para la extracción, visualización y análisis de redes de 
calles. Mediante esta librería se puede obtener toda la información referente a un 
espacio determinado, e importar toda la información de sus caminos y carreteras 
transitables. De esta forma tenemos toda una estructura de datos con información 
geográfica precisa del entorno. Por otra parte mediante esta librería se puede visualizar 
las trayectorias escogidas, así como almacenar imágenes de las trayectorias analizadas 
o simuladas.
%matplotlib
\item \textbf{matplotlib} Librería por excelencia para la representación de información 
de forma visual. Mediante el uso de esta librería podemos mostrar diversas gráficas de 
la información obtenida y analizada por los diferentes algoritmos.
%numpy
\item \textbf{numpy} Librería para el cálculo científico. Esta librería nos aporta 
diferentes estructuras de datos para el acceso y cálculo de forma eficiente a la 
información. Una de las principales ventajas de esta librería es la implementación de 
matrices de forma que el acceso a los datos incrementa su eficiencia de forma 
considerable.
%geopy
\item \textbf{geopy} Librería para el tratamiento de coordenadas. Mediante esta librería 
se obtienen las herramientas necesarias para tratar al par de datos \textit{(Lat,Long)} 
como un punto de coordenada  \ac{GPS}. Por otra parte se obtiene implementación del 
cálculo de la distancia entre dos coordenadas.
%itertools
\item \textbf{itertools} Esta librería implementa diversos bloques de iteradores. El uso 
dentro de nuestro proyecto queda exclusivamente para la agrupación de elementos 
dentro de un set de datos.
%sklearn
\item \textbf{sklearn} Librería para el análisis de de datos. Esta librería nos aporta todas 
las herramientas necesarias para el análisis de los datos \ac{GPS}. Con ella realizamos 
los cálculos de puntos próximos a partir de una búsqueda basada en \ac{BSP} como 
veremos en la siguiente sección.
%shapely
\item \textbf{shapely} Librería para el tratamiento de figuras geométricas. De esta 
forma podemos tratar elementos comos puntos \ac{GPS} de forma abstracta. Por otra 
parte nos permite tratar las rutas del espacio a analizar como una linea de sucesivos 
puntos \ac{GPS}.
\end{enumerate}

\subsection{Docker}
Para poder entender completamente este apartado tenemos que hacer una mención a 
la forma clásica de arquitectura de proyectos software a nivel de infraestructura. Esta 
forma tenía como particularidad el uso de máquinas virtuales para la ejecución de las 
diferentes aplicaciones, así como un supervisor de las máquinas virtuales.  Cada 
máquina virtual despliega su propio sistema operativo con sus dependencias. Por otro 
lado, encontramos la forma mediante el uso de contenedores.

Una de las implementaciones más usadas de contenedores es \textbf{Docker}. Docker 
es una plataforma para el desarrollo, despliegue y ejecución de aplicaciones basado en 
contenedores \cite{Docker01}.  Una imagen es una plantilla de lectura donde está 
almacenado todo el código (normalmente compilado) junto con todas las dependencias 
necesarias para que pueda ser ejecutado completamente. Un contenedor es una 
instancia de una imagen, que puede ser creada, desplazable y parada a partir de la 
\ac{API} de Docker.

\begin{figure}[!htb]
\begin{center}
\includegraphics[width=0.9\textwidth]{./Imagenes/DockerComparision.png}
\caption{Comparación entre arquitectura Docker y mediante máquinas virtuales. 
\cite{Docker02}}
\label{figure:DockerComparision}
\end{center}
\end{figure}

Con esta nueva forma, todas las aplicaciones comparten el mismo sistema operativo, 
no es necesario ningún supervisor y la responsabilidad de las dependencias queda 
localizada en los contenedores, que contienen únicamente lo necesario para poder 
ejecutar la aplicación \ref{figure:DockerComparision}.

La ejecución del aplicativo tiene como requerimientos una serie de dependencias, que 
la máquina física puede tener o no en su sistema. Para añadir una capa de abstracción 
a la arquitectura y hacerlo replicable y desplegable en cualquier máquina
se ha utilizado Docker. La ventaja de usar Docker en esta propuesta reside en la 
capacidad de la plataforma de usar los recursos del sistema de forma eficiente, así 
como la portabilidad que ofrece. La infraestructura propuesta se muestra en la figura 
\ref{figure:DockerStructure}:
\begin{figure}[htb]
\begin{center}
\includegraphics[width=0.8\textwidth]{./Imagenes/DockerStructure}
\caption{Infraestructura de la propuesta de \textit{track-simulator}}
\label{figure:DockerStructure}
\end{center}
\end{figure}
\newpage

Como vemos la aplicación consta de un \ac{CLI} que se encarga de centralizar todo 
el aplicativo. Mediante el uso de comandos se podrá realizar cada una de las acciones 
de la aplicación. Es este \ac{CLI} el encargado de desplegar los contenedores tanto de 
la base de datos como de la aplicación. Añade una serie de directorios que son el 
puente de comunicación entre la aplicación y la máquina que lo ejecuta. En una carpeta 
se añaden los ficheros \ac{GPX} a analizar, y como salida de la simulación aparecen 
ficheros \ac{GPX} y representaciones gráficas de la ruta en formato \ac{PNG}.