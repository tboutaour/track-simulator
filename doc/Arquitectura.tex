%!TeX root=MemoriaTFG.tex

\chapter{Arquitectura de la aplicación}

%Qué se puede hacer con TrackSimulator
En esta sección presentamos la arquitectura de la aplicación. Realizaremos una 
descripción de las funcionalidades que presenta y de los módulos de los que consta.

\section{\textit{TrackSimulator}}
\textit{TrackSimulator} es un generador pseudo-aleatorio de trayectorias geoposicionales.

Entendemos como simulación al proceso de recrear el comportamiento, en este caso el camino 
de usuarios dentro de un espacio geográfico. Como muestra esencial y discreta del comportamiento de la 
población se usan registros que han sido obtenidos mediante localización \ac{GPS}. En la figura X
se muestra un ejemplo de una detección del recorrido de un individuo.

\begin{figure}[htb]
\begin{center}
\includegraphics[width=0.6\textwidth]{./Imagenes/RealTrackDetection.png}
\caption{Ejemplo de detección real de la trayectoria de un individuo por la Serra de Tramuntana, Illes 
Balears, Mallorca, Spain}
\label{TrackExample1}
\end{center}
\end{figure}
Para realizar un algoritmo que genere trayectorias y tengan un grado de similitud con la realidad se debe 
realizar un ejercicio previo de análisis y tratamiento de información. Distancias entre las detecciones, 
desviación entre la detección del dispositivo \ac{GPS} y un camino definido son ejemplos de datos que se 
tienen que tener en cuenta para la obtención de resultados óptimos en cuanto a fidelidad a la realidad.

Tanto la información muestral como la generada, así como todo el conjunto de información geoespacial debe 
ser almacenada en un tipo de estructura lógica que permita el acceso y manipulación de estos de la forma 
más eficiente posible. 

Teniendo una estructura lógica que permita almacenar la información, el problema determinante será la 
forma en la que los datos geoposicionales, en este caso puntos \ac{GPS} pasan a formar parte de esté 
modelo. Recordemos que un punto \ac{GPS} contiene únicamente información de la posición dentro de la
superficie terrestre, no obstante no aporta información de la asignación de este punto a un segmento de
un camino, calle, o paraje concreto. Este problema tiene el nombre de \textit{Map matching} y la propuesta
de este documento a su resolución se realizará en detalle en el apartado \ref{section: MapMatching}.

Con la estructura lógica y la información integrada en ella, el análisis de la información permitirá tomar 
el conjunto de decisiones que permitan maximizar el grado de exactitud de la simulación para, 
posteriormente, realizar el algoritmo que permita realizar dicha recreación.

\begin{figure}[H]
\begin{center}
\includegraphics[width=0.5\textwidth]{./Imagenes/TrackSimulatorStructure.png}
\caption{Diagrama de funcionamiento de TrackSimulator}
\label{TrackSimulatorDiagram}
\end{center}
\end{figure}
 
La aplicación crea una sucesión de puntos que equivalen a un recorrido dentro de un plano geográfico. 
Se ha determinado que la aplicación realiza la generación de forma pseudo-aleatoria debido a que 
la producción de los diferentes puntos no sigue ningún patrón o regularidad.

La sucesión de puntos estará relacionada directamente con un camino, sendero o recorrido asignado al 
modelo lógico. La simulación tendrá en cuenta la frecuencia de paso relativa por el segmento como parte
del proceso, así como la distancia entre los puntos.


\section{Funcionalidades de \textit{TrackSimulator}}
\textit{TrackSimulator} es una aplicación. El término \textit{aplicación} se entiende como la suma de 
conjunto de implementaciones  de código, ejecutable y del que se espera un resultado. La aplicación 
permite:
\begin{description}
\item [Importación y análisis de archivos \ac{GPX}] La aplicación soportará la importación de un archivo GPX concreto, 
o bien la ruta de una carpeta con diversos archivos GPX. Estos archivos GPX deben contener trayectorias realizadas en 
el espacio delimitado.

La aplicación, con los datos importados por los archivos, permitirá un análisis de trayectorias aplicando técnicas de 
\textit{map-matching}. Del análisis se obtendrá información:
\begin{description} 
\item[Distancia entre puntos] Como la distancia entre las capturas de posición \ac{GPS} en el fichero.
\item[Distancia relativa entre punto de ruta y punto de trayectoria] Como la distancia del punto \ac{GPS} detectado 
de la trayectoria a la punto proyección dentro de la ruta.
\item[Frecuencia de paso por segmento de ruta] Como la frecuencia de paso por el segmento $x_{a}, x_{b}$ relativo 
a todos los segmentos desde $x_{a}$.
\end{description}


Toda esta información quedará almacenada en una base de datos de forma que la información sea accesible para realizar la simulación de la ruta.

\item [Muestra de resultados del análisis] La aplicación permite la impresión por pantalla de información gráfica de los resultados del análisis.
\item [Creación de una trayectoria a partir de parámetros] Con los resultados de la información analizada se puede realizar una simulación de trayectorias dentro del espacio geográfico.
\item [Exportación de trayectorias a fichero \ac{GPX}] La aplicación permite realizar una exportación de las trayectorias simuladas en formato GPX.
\item [Visualización de la trayectoria] La aplicación mostrará una representación gráfica de la trayectoria tanto analizada como simulada dentro del territorio geográfico.

\end{description}
\begin{figure}[htb]
\begin{center}
\includegraphics[width=0.9\textwidth]{./Imagenes/TrackSimulatorDiagram.png}
\caption{Diagrama de funcionamiento de TrackSimulator}
\label{TrackSimulatorDiagram}
\end{center}
\end{figure}

\section{Requerimientos de \textit{TrackSimulator}}
\label{section: RequerimientosTrackSimulator}
Explicadas las funcionalidades de la aplicación de Track Analyzer los requerimientos identificados son los siguientes:
\begin{description}
\item[Importación de datos] La aplicación debe realizar una importación de los datos desde un fichero .\ac{GPX} al modelo lógico elegido para la el posterior tratamiento.
\item[Análisis de ruta] La aplicación debe realizar un análisis de la ruta que proporcione por una parte indicadores y por otro valores medibles, cuantificables y utilizables para realizar una simulación 
lo más precisa posible.
\item[Almacenamiento de las rutas reales en base de datos] La aplicación debe realizar un almacenamiento de los datos analizados en una base de datos, con el objetivo de poder ser repetible
sin necesidad de importar nuevamente los datos.
\item[Almacenamiento del análisis en base de datos] La aplicación debe realizar un almacenamiento de los resultados del análisis en una base de datos, con el objetivo de poder ser accesibles para la
etapa de simulación.
\item[Muestra de gráficas del análisis de los datos] La aplicación debe poder realizar una muestra de la información resultante para su visualización por parte del usuario.
\item[Lectura de la base de datos para el acceso a la información] La aplicación debe poder realizar una lectura de la base de datos para acceder a la información necesaria para la muestra de resultados o 
simulación de trayectorias.
\item[Obtención del camino más probable a partir de parámetros de entrada] La aplicación debe poder generar el camino más probable dada un análisis previo y unos parámetros de entrada.
\item[Simulación de puntos \ac{GPS} a partir de resultado de análisis] La aplicación debe generar los puntos de cada uno de los segmentos necesarios dado un camino.
\item[Generación del fichero GPX correspondiente a la simulación de la trayectoria] La aplicación realizar una exportación de los datos a formato .\ac{GPX}.
\item[Visualicación de trayectoria] La aplicación debe poder mostrar al usuario una visualización de la trayectoria dado un fichero .\ac{GPX}.
\end{description}

\section{Implementación de TrackSimulation}
\subsection{Flujo de datos}

\begin{figure}[htb]
\begin{center}
\includegraphics[width=0.9\textwidth]{./Imagenes/TrackSimulatorDataFlow.png}
\caption{Flujo de datos de TrackSimulator}
\label{TrackSimulatorDataFlow}
\end{center}
\end{figure}

Podemos ver en la figura \ref{TrackSimulatorDataFlow} de la parte superior la implementación de la aplicación 
dos grandes módulos. Por una parte encontramos \textit{TrackAnalyzer}. Este módulo es el encargado de 
realizar todas los procesos de tratamiento de datos para su posterior análisis. Inicialmente en el alcance de 
este proyecto se realizará análisis de ficheros \ac{GPX} por lo que es el único modelo de datos que entrará en 
nuestro sistema. Para realizar un análisis se realiza un proceso de tratamiento previo de datos, que se detalla 
en el apartado \ref{section: ImportacionGPX}.

Posterior al tratamiento de datos se realizará el proceso de \textit{Map Matching} con el que se une los datos 
introducidos al modelo lógico del territorio geográfico. El proceso de \textit{Map Matching} queda descrito 
en detalle en el apartado \ref{section: MapMatching}.

Una vez se ha realizado la asignación por cada punto a un camino determinado se puede realizar un análisis 
del fichero, del cual se obtienen las métricas requeridas por el apartado 
\ref{section: RequerimientosTrackSimulator}. La descripción de la forma en la que se explotan los datos está
detallada en la sección \ref{section:ExplotacionDato}.

Con la explotación del dato hecha, únicamente queda guardar la información generada en una base de datos.
El almacenamiento de las trayectorias analizadas y mapeadas, el resultado de los análisis y la estructura 
lógica con la información modificada se realizarán de forma independiente en secciones diferentes de la 
base de datos.

El módulo de \textit{TrackSimulator} tiene la responsabilidad de generar la simulación de una trayectoria 
a partir de unos parámetros de entrada. Una vez generada la simulación, un fichero \ac{GPX} con las 
coordenadas será el producto final del proceso.
 
\subsection{Herramientas externas utilizadas}
Como herramientas externas utilizadas para la realizar la propuesta de este documento se destacan 
el tipo de almacenamiento de datos que se utilizarán, así como las librerías externas que se han usado 
para la implementación de la aplicación. A continuación se detallan ambas.

\subsubsection{Almacenamiento del dato}
Tanto de trayectorias reales como de los datos generados por el análisis deben ser almacenados 
de forma que sean accessibles de forma rápida.
Para el almacenamiento de datos, existen dos grandes posibilidades: El uso de Bases de datos relacionales, 
a partir de ahora descritas como \ac{SQLDB},
o el opuesto, las bases de datos no relacionales, \ac{NOSQLDB}.
Las ventajas de las \ac{NOSQLDB} han hecho que con el paso del tiempo, los desarrollos se vean orientados 
al tipo de almacenamiento no relacional.
Para la realización de la propuesta descrita en este documento se ha decidido realizar el almacenamiento de 
toda la información en \textit{MongoDB}, una de las \ac{NOSQLDB}
más usadas y que detallamos a continuación.

\textit{MongoDB} se trata de una \ac{NOSQLDB} de código abierto y su funcionamiento es documental. 
Se almacenan colecciones de documentos, que son series de elementos JSON clave-valor.

MongoDB elimina las limitaciones de las bases de datos relacionales \cite{Mongo01}. 
Permite almacenar de forma eficiente grandes cantidades de información, siendo
a su vez flexible a modificaciones debido a que los documentos que se almacenan en las colecciones no tienen 
una taxonomía definida. Por lo que el desarrollo incremental de la aplicación y la aparición de nuevos 
campos dentro del modelo de datos no es un problema.
Otro gran motivo para la elección de MongoDB como base de datos de este proyecto es la escalabilidad 
que ofrece, de forma que a grandes cantidades de trayectorias por almacenar, la aplicación respondería 
de forma eficiente y mucho más precisa.

Actualmente para el desarrollo de la propuesta no se ha contado con una gran cantidad de datos.
No obstante el problema ha sido planteado con el objetivo de poder analizar grandes cantidades de datos y que puedan
ser almacenados y accesible mediante el uso de esta base de datos.
\subsubsection{Librerías utilizadas}
Para la realización de las funcionalidades comentadas en la parte anterior se ha hecho uso de las siguientes librerías externas:
\begin{description}
%gpxpy
\item[gpxpy] Librería para la manipulación de ficheros \ac{GPX}. Mediante esta librería se puede realizar una importación del fichero para su posterior manipulación. De esta forma se obtiene la secuencia de puntos \ac{GPS} que describe una trayectoria determinada.
%pandas
\item[pandas] Librería para el análisis de los datos. Toda la información correspondiente a las diferentes trayectorias queda reflejada en un Dataframe para su posterior tratamiento y acceso.
%osmnx
\item[osmnx] Librería para la extracción, visualización y análisis de redes de calles. Mediante esta librería se puede obtener toda la información referente a nuestro espacio determinado del Castillo de Bellver e importar toda la información de sus caminos y carreteras transitables. De esta forma tenemos toda una estructura de datos con información geográfica precisa del entorno. Por otra parte mediante esta librería se puede visualizar las trayectorias escogidas, así como la capacidad de almacenar imágenes de las trayectorias analizadas o simuladas.
%matplotlib
\item[matplotlib] Librería por excelencia para la representación de información de forma visual. Mediante el uso de esta librería podemos mostrar diversas gráficas de la información obtenida y analizada por los diferentes algoritmos.
%numpy
\item[numpy] Librería para el cálculo científico. Esta librería nos aporta diferentes estructuras de datos para el acceso y cálculo de forma eficiente a la información. Una de las principales ventajas de esta librería es la implementación de matrices de forma que el acceso a los datos incrementa su eficiencia de forma considerable.
%geopy
\item[geopy] Librería para el tratamiento de coordenadas. Mediante esta librería se obtienen las herramientas necesarias para tratar al par de datos \textit{(Lat,Long)} como un punto de coordenada  \ac{GPS}. Por otra parte se obtiene implementación del cálculo de la distancia entre dos coordenadas.
%itertools
\item[itertools] Esta librería implementa diversos bloques de iteradores. El uso dentro de nuestro proyecto queda exclusivamente para la agrupación de elementos dentro de un set de datos.
%sklearn
\item[sklearn] Librería para el análisis de de datos. Esta librería nos aporta todas las herramientas necesarias para el análisis de los datos \ac{GPS}. Con ella realizamos los cálculos de puntos próximos a partir de una búsqueda basada en \ac{BSP} como veremos en la siguiente sección.
%shapely
\item[shapely] Librería para el tratamiento de figuras geométricas. De esta forma podemos tratar elementos comos puntos \ac{GPS} de forma abstracta. Por otra parte nos permite tratar las rutas del espacio a analizar como una linea de sucesivos puntos \ac{GPS}.
%pickle
\item[pickle] Librería para la codificación-decodificación de ficheros. Mediante el uso de esta librería se realiza la exportación de la información estadística y de la estructura de grafo generada y almacenada en los ficheros .txt/edgelist sucesivamente.
\end{description}