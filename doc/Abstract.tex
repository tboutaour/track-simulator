%!TeX root=MemoriaTFG.tex

\chapter{Abstract}

El auge de la explotación de los datos para obtener ventajas estratégicas de negocio abarca gran parte de los ámbitos de la sociedad. El aumento de dispositivos capaces de  captar la ubicación de un individuo mediante \ac{GIS} sumado al aumento de dispositivos inteligentes que los usuarios portan durante el día hace que la  información de posicionamiento geográfico sea valiosa en un gran número de sectores. 

\textit{\textbf{TrackSimulator}} es el \ac{CLI} desarrollado en Python propuesto en el documento que permite la generación de trayectorias de forma pseudo-aleatoria. La aplicación consta de un módulo de análisis de de ficheros en formato \ac{GPX}, de los que se obtiene información medible y cuantitativa que es usada para generar trayectorias. Esta herramienta almacena en una base de datos el análisis del conjunto de trayectorias reales que se desee. Del proceso de análisis se obtiene una representación gráfica del comportamiento de las trayectorias. De la simulación se obtiene un fichero en \ac{GPX} y una representación gráfica por cada una de las trayectorias generadas.

Este documento explica en detalle la solución propuesta. Se muestra la arquitectura de la herramienta, el flujo de datos, así como las heurísticas de análisis y los algoritmos empleados para el análisis y la simulación de trayectorias, demostrando mediante experimentación los resultados que pueden obtenerse.
