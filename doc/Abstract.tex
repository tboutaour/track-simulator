%!TeX root=MemoriaTFG.tex

\chapter*{Abstract}
El aumento, en la actualidad, de dispositivos capaces de captar la ubicación de un 
individuo mediante \ac{GIS} ha permitido ampliar el volumen de fuentes de datos sobre la 
geolocalización de personas. 

En este trabajo, se presenta el proyecto denominado \textit{\textbf{track-simulator}}. Se 
trata de un \ac{CLI} desarrollado en Python que permite la generación de trayectorias de 
forma \textbf{pseudo-aleatoria}. Esta generación se realiza en base a un módulo de 
análisis previo, que trabaja con trayectorias reales a partir de ficheros \textbf{\ac{GPX}}. 
Mediante el uso un modelo estadístico basado en cadenas de Markov, \ac{HMM}, se 
propone una solución al problema de la asignación de puntos geoposicionales a una red 
de carreteras. Con esto se permite relacionar coordenadas geográficas a una estructura 
lógica de vías y carreteras, analizable y explotable.

Esta propuesta sostiene un mecanismo que permite la producción de trayectorias dentro 
de un territorio geográfico, con un comportamiento aproximado a las trayectorias reales 
introducidas al aplicativo. La generación de esta información es útil en todo sector que 
esté interesado en el análisis de comportamiento de individuos dentro de un territorio 
delimitado. Vías de paso frecuente, caminos poco transitados o zonas inaccesibles a 
usuarios son, entre otros, los indicadores que pueden ser obtenidos mediante la 
explotación de esta herramienta.

Este documento explica en detalle la solución propuesta. Se muestra la arquitectura de la 
herramienta, el flujo de datos, así como las heurísticas y los algoritmos empleados para el 
análisis y la simulación de trayectorias, demostrando mediante experimentación, los 
resultados que pueden obtenerse con el uso de este aplicativo.