%!TeX root=MemoriaTFG.tex

\chapter{Introducción}

En plena era del dato \cite{Borkovich01}, la necesidad por obtener conocimiento a partir de gran cantidad de datos generados ha hecho que la cantidad de avances en este ámbito incremente notablemente. En la actualidad tenemos toda una serie de dispositivos: smartphones, smartwatches, pulseras deportivas, chalecos deportivos que están generando información diversa del individuo que lo usa. La posición geográfica es una de las principales, seguida de información temporal, frecuencia cardíaca y un conjunto de información que puede ser tratada para obtener conocimiento sobre la población. 

La información del posicionamiento geográfico  de los individuos que los \ac{GIS} proporcionan constantemente hace que sea posible obtener, tratar, analizar y explotar dicha información. La explotación de estos datos pueden llevarse a diferentes aplicaciones. Puede ser usada para determinar la posición de un individuo, para sistemas de navegación en los que un individuo se desplaza desde un punto origen a un punto destino , la monitorización del movimiento de individuos, entre otras \cite{GPS01}. 

Una de las grandes utilidades de la explotación de estos datos consiste en el análisis de redes complejas de caminos y carreteras de forma que se puede obtener indicadores sobre el volumen de paso de individuos por caminos o carreteras, así como para la planificación de trayectorias. Un ejemplo claro de esta utilidad es la explotación del dato para el conocimiento en el transporte.  Tanto en la forma en la que los individuos se trasladan al análisis de las carreras de un vehículo de transporte, tanto público como privado, como Uber o Cabify \cite{Viskic01}.  

Es en este ámbito en el que aparece la necesidad de una herramienta que sea capaz de analizar información que obtienen los diferentes \ac{GIS}, almacenarla y transformar este conocimiento para poder generar trayectorias similares a la realidad dentro de un territorio geográfico.

TrackSimulator es un \ac{CLI} para la generación de trayectorias pseudo-aleatorias propuesta en este documento. TrackSimulator cumple una serie de requerimientos que están descritos detalladamente en el capítulo \ref{chapter:AppArchitecture}. Estos requerimientos son:
\begin{enumerate}

\item Importación de datos en formato \ac{GPX}.

\item Análisis de trayectorias, obteniendo información medible y cuantificable.

\item Almacenamiento de las rutas reales y de los resultados del análisis en base de datos

\item Generación de fichero de gráficas resultantes del análisis 

\item Lectura de la base de datos para el acceso a la información

\item Simulación de puntos \ac{GPS}

\item Generación del fichero GPX correspondiente a la simulación de la trayectoria

\item Generación de fichero visualización de trayectoria

\end{enumerate} 

Esta propuesta está desarrollada en el lenguaje Python y utiliza diferentes librerías externas, entre las que se destaca \textbf{\textit{osmnx}} para el el análisis a partir del uso de redes de caminos y carreteras \cite{Boeing01}, \textbf{\textit{pandas}} para el tratamiento de datos \cite{Pandas01} y \textbf{\textit{gpxpy}} para la manipulación de ficheros \ac{GPX} \cite{Gpxpy01}. El resto de librerías externas utilizadas se describen en detalle en el capítulo \ref{chapter:AppArchitecture}

La infraestructura de la aplicación está formada a partir de contendores de Docker, de esta forma, la aplicación puede ser ejecutada en cualquier máquina que tenga Docker instalado, eliminándose cualquier tipo de problema que pueda surgir entre dependencias del aplicativo y la máquina en la que se ejecuta. Encontramos entonces dos contenedores: el primero contiene una conexión con una base de datos MongoDB en la que almacena tanto las trayectorias analizadas como los resultados de los diferentes análisis. El segundo contenedor contiene el aplicativo \textit{TrackSimulator}. Este contenedor contiene todo el código, dependencias y configuración necesario para que el usuario pueda analizar un conjunto de trayectorias almacenadas en ficheros  \ac{GPX}. Este análisis lo hace a partir de una técnica llamada \textit{map-matching} consistente en unir los datos con la red lógica de caminos y carreteras.

Despues de la etapa de análisis, el usuario puede realizar una generación de trayectorias. Estas son generadas de forma pseudo-aleatoria a partir de la información analizada en la anterior etapa. Cada trayectoria generada es única en si misma y replica el comportamiento de los individuos analizados.

Finalmente como productos resultantes de esta herramienta encontramos ficheros con la información resultante del análisis del conjunto de trayectorias, así como ficheros \ac{GPX} y representaciones gráficas de las trayectorias simuladas.

El uso de esta herramienta se hace a partir de la linea de comandos. Se usa una nomenclatura de facil entendimiento que sigue el formato que se ve en el comando \ref{comand:example}

\begin{lstlisting}[caption={Ejecución simulación}, language=bash, label={comand:example}]  
	ts-cli COMAND --PARAMETERS
\end{lstlisting}

Tanto la instalación como los comandos soportados por el aplicativo están descritos en el capítulo \ref{chapter:GuiaUso}

El objetivo de este documento es explicar al lector la propuesta tanto a nivel teórico como a nivel de implementación de código, así como guiar al usuario a poder instalar el aplicativo y probarlo. Este documento se compone de 9 capítulos donde aparecen 3 diferentes bloques: el primer bloque  (capítulo \ref{chapter:PrincipalConcepts} y \ref{chapter:AppArchitecture}   sería introductorio al problema y la arquitectura de la herramienta. Del capítulo \ref{chapter:DataAnalysis} al \ref{chapter:Experimentation} se explican los diferentes modulos que componen el aplicativo y el flujo de los datos, explicando detalladamente el análisis de los datos y el algoritmo de simulación de las trayectorias, cerrando el bloque con una muestra de los resultados la experimentación del aplicativo. Finalmente el capítulo \ref{chapter:GuiaUso} explica al lector como instalar y usar la herramienta. El capítulo \ref{chapter:Conclusion} cierra el documento con las opiniones final.

Los capítulos son los siguientes:

\begin{enumerate}[label={C. \arabic*.}]
\item \textbf{Conceptos principales}. Capítulo introductorio al lector. Aparecen los conceptos básicos y las herramientas conocidas dentro del marco del tratamiento de datos \ac{GPS}.

\item \textbf{Arquitectura de la aplicación}. Descripción de las funcionalidades que componen la propuesta. Flujo de datos de la herramienta y requerimientos básicos que cumple. Se muestra al lector la estructura de TrackSimulator y el problema que resuelve.

\item \textbf{Análisis de los datos \ac{GPS}}. Capítulo donde se explica detalladamente como se analiza el dato. Desde la preparación del entorno hasta la explotación del dato pasando detalladamente por el proceso de definición de heurísticas y parámetros.


\item \textbf{Simulación de trayectorias}. Explicación del proceso de generación de trayectorias pseudo-aleatorias. De define la forma en la que se generan individualmente los puntos \ac{GPS}. Se muestra la implementación técnica destacable.

\item \textbf{Experimentación}. Muestra de resultados del uso de la herramienta. Se realizan dos experimentaciones: con introducción de datos analizados y sin introducción de datos. Se muestran gráficas comparativas y se comentan en detalle los resultados.

\item \textbf{Guía de instalación y uso}. Conjunto de pasos a seguir para instalar el aplicativo en una máquina. En ese capítulo se explican instrucciones disponibles y los parámetros necesarios para su ejecución, completando con ejemplos.

\item \textbf{Conclusión}. Capítulo de cierre del documento y de la propuesta. Se describe el futuro del aplicativo, así como la opinión personal del desarrollo del proyecto. Se concluye la propuesta.
\end{enumerate}


